% !TeX encoding = UTF-8
% !TeX spellcheck = pl_PL

% $Id:$

%Author: Wojciech Domski
%Szablon do ząłożeń projektowych, raportu i dokumentacji z steorwników robotów
%Wersja v.1.0.0
%

%% Konfiguracja:
\newcommand{\kurs}{Sterowniki robot\'{o}w}
\newcommand{\formakursu}{Projekt}

%odkomentuj właściwy typ projektu, a pozostałe zostaw zakomentowane
%\newcommand{\doctype}{Za\l{}o\.{z}enia projektowe} %etap I
\newcommand{\doctype}{Raport} %etap II
%\newcommand{\doctype}{Dokumentacja} %etap III

%wpisz nazwę projektu
\newcommand{\projectname}{Automatyczny barman Dionizos}

%wpisz akronim projektu
\newcommand{\acronim}{AuBaDi}

%wpisz Imię i nazwisko oraz numer albumu
\newcommand{\osobaA}{Kewin \textsc{Gałuszka}, 241624}
\newcommand{\osobaB}{Adrian \textsc{Urban}, 241558}
%w przypadku projektu jednoosobowego usuń zawartość nowej komendy


%wpisz termin w formie, jak poniżej dzień, parzystość, godzina
\newcommand{\termin}{ptTP11}

%wpisz imię i nazwisko prowadzącego
\newcommand{\prowadzacy}{dr in\.{z}. Wojciech \textsc{Domski}}

\documentclass[10pt, a4paper]{article}
\usepackage{float}
\usepackage{polski}
\usepackage{breakurl}
\usepackage[hyphens]{url}
\usepackage{hyperref}
\usepackage[utf8]{inputenc}
\usepackage{lscape}
\usepackage{url}

\include{preambula}

\begin{document}

\def\tablename{Tabela}	%zmienienie nazwy tabel z Tablica na Tabela

\begin{titlepage}
	\begin{center}
		\textsc{\LARGE \formakursu}\\[1cm]		
		\textsc{\Large \kurs}\\[0.5cm]		
		\rule{\textwidth}{0.08cm}\\[0.4cm]
		{\huge \bfseries \doctype}\\[1cm]
		{\huge \bfseries \projectname}\\[0.5cm]
		{\huge \bfseries \acronim}\\[0.4cm]
		\rule{\textwidth}{0.08cm}\\[1cm]
		
		\begin{flushright} \large
		\emph{Skład grupy:}\\
		\osobaA\\
		\osobaB\\[0.4cm]
		
		\emph{Termin: }\termin\\[0.4cm]

		\emph{Prowadzący:} \\
		\prowadzacy \\
		
		\end{flushright}
		
		\vfill
		
		{\large \today}
	\end{center}	
\end{titlepage}

\newpage
\tableofcontents
\newpage

%Obecne we wszystkich dokumentach
\section{Opis projektu}
\label{sec:OpisProjektu}

Celem projektu jest stworzenie automatycznego barmana -- urządzenia będącego w stanie mieszać płyny w ściśle zadanych proporcjach. Urządzenie wyposażone w wyświetlacz LCD oraz przyciski będzie oferować interfejs, który pozwoli użytkownikowi na wybranie konkretnego napoju. W zależności od decyzji użytkownika dotyczącej rodzaju napoju, urządzenie zmieni swoje podświetlenie wykonane za pomocą diod LED. Dodatkowo wykorzystany zostanie DAC do wygenerowania wcześniej zapisanego na pamięć dźwięku. Karetka, w której umieszczony zostanie zbiornik na płyn będzie przesuwana za pomocą silnika krokowego. Kolejne płyny będą pobierane z zasobników za pomocą indywidualnych pomp bądź zaworów.

\begin{figure}[H]
	\centering
	\includegraphics[width=0.7\textwidth]{figures/dionizos.png}
	\caption{Schematyczny rysunek urządzenia}
	\label{fig:Architektura}
\end{figure}

%Obecne we wszystkich dokumentach
\section{Konfiguracja mikrokontrolera}
 
Konfiguracja portów mikrokontrolera (rysunek 2) oraz zegarów (rysunek 3) przy użyciu prgoramu STM32CubeMX.
 
\begin{figure}[H]
	\centering
	\includegraphics[width=0.6\textwidth]{konfiguracja.PNG}
	\caption{Konfiguracja wyjść mikrokontrolera w programie STM32CubeMX}
	\label{fig:KonfiguracjaMikrokontrolera}
\end{figure}

\newpage
\begin{figure}[H]
	\centering
	\includegraphics[width=0.9\textheight,angle=90]{zegary.PNG}
	\caption{Konfiguracja zegarów mikrokontrolera}
	\label{fig:KonfiguracjaZegara}
\end{figure}

%Obecne we wszystkich dokumentach
\subsection{Konfiguracja pinów}

\begin{table}[H]
	\centering
	\begin{tabular}{|l|l|l|l|}
		\hline
		Numer pinu	&	PIN & Tryb pracy & Funkcja/etykieta\\
		\hline
12&	PH0 - OSC{\_}IN&	RCC{\_}OSC{\_}IN&	PH0-OSC{\_}IN \\
13&	PH1 - OSC{\_}OUT&	RCC{\_}OSC{\_}OUT&	PH1-OSC{\_}OUT\\
17&	PC2&	GPIO{\_}Output&	Direction [A4988]\\
18&	PC3&	GPIO{\_}Output&	Step [A4988]\\
23&	PA0-WKUP&	ADC1{\_}IN0&	Laser{\_}receiver\\
24&	PA1&	ADC1{\_}IN1&	Endstop{\_}receiver\\
29&	PA4&	I2S3{\_}WS&	I2S3{\_}WS [CS43L22{\_}LRCK]\\
30&	PA5&	GPIO{\_}Input&	Button{\_}1\\
31&	PA6&	GPIO{\_}Input&	Button{\_}2\\
32&	PA7&	GPIO{\_}Input&	Button{\_}3\\
47&	PB10&	I2C2{\_}SCL&	I2C2{\_}SCL{\_}LCD\\
55&	PD8&	GPIO{\_}Output&	LED{\_}blue\\
56&	PD9&	GPIO{\_}Output&	LED{\_}green\\
57&	PD10&	GPIO{\_}Output&	LED{\_}red\\
58&	PD11&	GPIO{\_}Output&	Pump{\_}relay{\_}5\\
59&	PD12&	GPIO{\_}Output&	Pump{\_}relay{\_}4\\
60&	PD13&	GPIO{\_}Output&	Pump{\_}relay{\_}3\\
61&	PD14&	GPIO{\_}Output&	Pump{\_}relay{\_}2\\
62&	PD15&	GPIO{\_}Output&	Pump{\_}relay{\_}1\\
64&	PC7&	I2S3{\_}MCK&	I2S3{\_}MCK [CS43L22{\_}MCLK]\\
68&	PA9&	USB{\_}OTG{\_}FS{\_}VBUS&	\\
70&	PA11&	USB{\_}OTG{\_}FS{\_}DM&	OTG{\_}FS{\_}DM\\
71&	PA12&	USB{\_}OTG{\_}FS{\_}DP&	OTG{\_}FS{\_}DP\\
78&	PC10&	I2S3{\_}CK&	I2S3{\_}SCK [CS43L22{\_}SCLK]\\
80&	PC12&	I2S3{\_}SD&	I2S3{\_}SD [CS43L22{\_}SDIN]\\
85&	PD4&	GPIO{\_}Output&	Audio{\_}RST [CS43L22{\_}RESET]\\
86&	PD5&	GPIO{\_}Input&	OTG{\_}FS{\_}OverCurrent\\
89&	PB3&	I2C2{\_}SDA&	I2C2{\_}SDA{\_}LCD\\
92&	PB6&	I2C1{\_}SCL&	Audio{\_}SCL [CS43L22{\_}SCL] \\
96&	PB9&	I2C1{\_}SDA&	Audio{\_}SDA [CS43L22{\_}SDA]\\

		\hline
	\end{tabular}
	\caption{Konfiguracja pinów mikrokontrolera}
	
\end{table}
\begin{itemize}
\item Piny PD11,PD12,PD13,PD14,PD15 - załączanie przekaźników pomp
\item Piny PB6 i PB9 - komunikacja I2C z DAC Audio
\item Piny PA4, PC7, PC10, PC12 - komunikacja I2S z DAC Audio
\item Pin PD4 - Reset DAC Audio
\item Piny PD8, PD9, PD10 - oświetlenie LED
\item Piny PA5, PA6, PA7 - przyciski
\item Piny PC2, PC3 - sterownik silnika krokowego A4988
\item Piny PH0, PH1 - rezonator kwarcowy
\item Piny PA0, PA1 - czujniki fotorezystorowe (wyłącznik krańcowy oraz czujnik zbiornika)
\item Piny PB10, PB3 - komunikacja I2C z wyświetlaczem LCD
\item Piny PA9, PA11, PA12, PD5 - USB OTG (host)
\end{itemize}



%Obecne we wszystkich dokumentach
\subsection{I2C1}

Magistrala szeregowa I2C1 zostanie wykorzystana do komunikacji z układem CS43L22 -- DAC Audio

\begin{table}[H]
	\centering
	\begin{tabular}{|l|c|} \hline
		\textbf{Parametr} & Wartość \\
		\hline
		\hline  \textbf{I2C Speed Mode}& Standard Mode \\  \hline
		\textbf{I2C Clock Speed (Hz) } & 100000 \\
		
		\hline  \textbf{Primary Address Length selection}& 7-bit  \\\hline

	\end{tabular}
	\caption{Konfiguracja peryferium I2C1}
	\label{tab:USART}
\end{table}
\subsection{I2C2}

Magistrala szeregowa I2C2 zostanie wykorzystana do komunikacji z kontrolerem wyświetlacza HD44780 

\begin{table}[H]
	\centering
	\begin{tabular}{|l|c|} \hline
		\textbf{Parametr} & Wartość \\
		\hline
		\hline  \textbf{I2C Speed Mode}& Standard Mode \\  \hline
		\textbf{I2C Clock Speed (Hz) } & 100000 \\
		
		\hline  \textbf{Primary Address Length selection}& 7-bit  \\\hline

	\end{tabular}
	\caption{Konfiguracja peryferium I2C2}
	\label{tab:USART}
\end{table}
\subsection{I2S3}

Magistrala szeregowa I2S3 zostanie wykorzystana do komunikacji z układem CS43L22 -- DAC Audio 

\begin{table}[H]
	\centering
	\begin{tabular}{|l|c|} \hline
		\textbf{Parametr} & Wartość \\
		\hline
		\hline  \textbf{Transmission Mode}& Mode Master Transmit \\ 
		\hline  \textbf{Communication Standard} & I2S Philips \\
		\hline  \textbf{Data and Frame Format} & I2S 16 Bits Data on 16 Bits Frame \\
		\hline  \textbf{Selected Audio Frequency} & \textcolor{blue}{48KHz }\\
		\hline  \textbf{Real Audio Frequency} & \textcolor{blue}{48.076KHz} \\
		\hline  \textbf{Error between Selected and Real} & \textcolor{blue}{0.15{\%}}
		\\
	\hline  \textbf{Clock Source} & I2S PLL Clock  \\
	\hline  \textbf{Clock Polarity} & Low \\
	\hline

	\end{tabular}
	\caption{Konfiguracja peryferium I2S3}
	\label{tab:USART}
\end{table}



\subsection{USB OTG}

USB OTG zostanie wykorzystane do podłączenia zewnętrznej pamięci flash

\begin{table}[H]
	\centering
	\begin{tabular}{|l|c|} \hline
		\textbf{Parametr} & Wartość \\
		\hline
\hline  \textbf{Mode:} &Host{\_}Only\\
\hline  \textbf{mode:} &Activate{\_}VBUS \\
\hline  \textbf{Speed Host Full Speed} &12MBit/s\\
\hline  \textbf{Signal start of frame} &Disabled\\
 \hline 
	\end{tabular}
	\caption{Konfiguracja peryferium USB OTG}
	\label{tab:USART}
\end{table}
\newpage
\subsection{FATFS}
 Obsługa systemu plików FAT

\begin{table}[H]
	\centering
	\begin{tabular}{|l|c|} \hline
		\textbf{Parametr} & Wartość \\
		\hline
	\hline  \textbf{Function Parameters:}\\
\hline  \textbf{FS{\_}READONLY (Read-only mode)} & Disabled \\
\hline  \textbf{FS{\_}MINIMIZE (Minimization level)} &  Disabled\\
\hline  \textbf{USE{\_}STRFUNC (String functions)} &  Enabled with LF -> CRLF conversion\\
\hline  \textbf{USE{\_}FIND (Find functions) } & Disabled\\
\hline  \textbf{USE{\_}MKFS (Make filesystem function)} &  Enabled\\
\hline  \textbf{USE{\_}FASTSEEK (Fast seek function)} &  Enabled\\
\hline  \textbf{USE{\_}EXPAND (Use f{\_}expand function) } & Disabled\\
\hline  \textbf{USE{\_}CHMOD (Change attributes function)} &  Disabled\\
\hline  \textbf{USE{\_}LABEL (Volume label functions)} &  Disabled\\
\hline  \textbf{USE{\_}FORWARD (Forward function) } & Disabled\\
\hline  \textbf{Locale and Namespace Parameters:}\\
\hline  \textbf{CODE{\_}PAGE (Code page on target)} &  Latin 1\\
\hline  \textbf{USE{\_}LFN (Use Long Filename)} &  Disabled\\
\hline  \textbf{MAX{\_}LFN (Max Long Filename)} &  255\\
\hline  \textbf{LFN{\_}UNICODE (Enable Unicode)} &  ANSI/OEM\\
\hline  \textbf{STRF{\_}ENCODE (Character encoding) } & UTF-8\\
\hline  \textbf{FS{\_}RPATH (Relative Path) } & Disabled\\
\hline  \textbf{Physical Drive Parameters:}  \\
\hline  \textbf{VOLUMES (Logical drives)} &  1\\
\hline  \textbf{MAX{\_}SS (Maximum Sector Size)} &  512\\
\hline  \textbf{MIN{\_}SS (Minimum Sector Size)} &  512\\
\hline  \textbf{MULTI{\_}PARTITION (Volume partitions feature)} &  Disabled\\
\hline  \textbf{USE{\_}TRIM (Erase feature)} &  Disabled\\
\hline  \textbf{FS{\_}NOFSINFO (Force full FAT scan)} &  0\\
\hline  \textbf{System Parameters:} \\
\hline  \textbf{FS{\_}TINY (Tiny mode) } & Disabled\\
\hline  \textbf{FS{\_}EXFAT (Support of exFAT file system)} &  Disabled\\
\hline  \textbf{FS{\_}NORTC (Timestamp feature)} &  Dynamic timestamp\\
\hline  \textbf{FS{\_}REENTRANT (Re-Entrancy)} &  Disabled\\
\hline  \textbf{FS{\_}TIMEOUT (Timeout ticks) } & 1000\\
\hline  \textbf{FS{\_}LOCK (Number of files opened simultaneously)} &  2\\

	
	\hline

	\end{tabular}
	\caption{Konfiguracja peryferium I2S3}
	\label{tab:USART}
\end{table}
\newpage
\subsection{ADC}

ADC zostanie wykorzystany do odczytywania stanu z czujników optycznych

\begin{table}[H]
	\centering
	\begin{tabular}{|l|c|} \hline
		\textbf{Parametr} & Wartość \\
 \hline 

\hline  \textbf{ADC{\_}Settings:}\\
\hline  \textbf{Clock Prescaler} &PCLK2 divided by 8 *\\
\hline  \textbf{Resolution }&12 bits (15 ADC Clock cycles)\\
\hline  \textbf{Data Alignment} &Right alignment\\
\hline  \textbf{Scan Conversion Mode}& Disabled\\
\hline  \textbf{Continuous Conversion Mode} &Disabled\\
\hline  \textbf{Discontinuous Conversion Mode}& Disabled\\
\hline  \textbf{DMA Continuous Requests}& Disabled\\
\hline  \textbf{End Of Conversion Selection}& EOC flag at the end of single channel conversion\\
\hline  \textbf{ADC{\_}Regular{\_}ConversionMode:}\\
\hline  \textbf{Number Of Conversion}& 1\\
\hline  \textbf{External Trigger Conversion Source }&Regular Conversion launched by software\\
\hline  \textbf{External Trigger Conversion Edge}& None\\
\hline  \textbf{Rank} &1\\
\hline  \textbf{Channel }Channel &0\\
\hline  \textbf{Sampling Time} &28 Cycles *\\
 \hline 
	\end{tabular}
	\caption{Konfiguracja peryferium ADC}
	\label{tab:USART}
\end{table}

%Obecne w dokumencie do etapu II oraz III
\section{Urządzenia zewnętrzne}

Rozdział ten powinien zawierać opis i konfigurację wykorzystanych ukladów
zewnętrznych, jak np. akcelerometr.

%Obecne w dokumencie do etapu II oraz III
\subsection{Akcelerometr -- LSM303C}

Akcelerometr został wykorzystany do ...

Konfiguracja rejestrów czujnika została zaprezentowana w ...
Wpisanie tych wartości do rejestrów urządzenia ... powoduje ...

\begin{table}[H]
	\centering
	\begin{tabular}{|l|c|} \hline
		\textbf{Rejestr} & Wartość \\
		\hline
		\hline
		CTRL\_REG2 (0x21) & 0x12\\\hline
		CTRL\_REG3 (0x22) & 0x13\\\hline
	\end{tabular}
	\caption{Konfiguracja peryferium USART}
	\label{tab:Akcelerometr}
\end{table}

%Obecne w dokumencie do etapu II oraz III
%Obecne w dokumencie do etapu II oraz III
\section{Projekt elektroniki}



W przypadku, w którym projekt uwzględnia zastosowanie 
dodatkowej elektroniki to wówczas jej opis powinien znaleźć się tutaj.
Należy dołączyć schematy elektroniczne w formacie PDF 
jako dodatek do dokumentu 
za pomocą \textit{include}. Również w przypadku wytworzenia 
płytek PCB powinny znaleźć się tutaj ich widoki za zachowaniem skali.
Można również dołączyć zdjęcia 
elektroniki po uprzednim skompresowaniu, aby wynikowy rozmiar 
skompilowanego dokumentu nie był za duży.
%Obecne w dokumencie do etapu II oraz III
\section{Konstrukcja mechaniczna}

\subsection{Podstawa}
Jako podstawę do stworzenia urządzenia wykorzystano część mechaniki z drukarki HP C1440. Usunięto zbędne elementy wyposażenia, pozostawiono między innymi wałek, karetkę i część elementów konstrukcyjnych.


\subsection{Konserwacja}
Z uwagi na zużycie paska klinowego dokonano jego wymiany na nowy. Wyczyszczono wszystkie elementy ruchome urządzenia oraz użyto oleju do maszyn celem zmiejszenia oporów ruchu karetki po wałku. Dokonano również korekcji napięcia paska klinowego. 

\subsection{Zmiany w konstrukcji}
W drukarce do poruszania karetką wykorzystany był silnik prądu stałego, który został wymieniony na silnik krokowy. Z uwagi na różnicę w średnicach osi obu silników dokonano rozwiercenia zębatki, którą umieszczono na wale silnika krokowego. Do konstrukcji przykręcony został wyłącznik krańcowy.
Odcięto zbędne elementy plastikowe utrzymujące ruchome element.




%Obecne w dokumencie do etapu II oraz III
\section{Opis działania programu}

Program służy do testownia implementacji alogrytmu poruszania karetką. W związku z wczesną fazą powstawania programu parametry pozycji, do której ma przyjechać karetka, ustawiane są bezpośrednio w kodzie. Przy starcie programu silnik krokowy wykonuje tyle skoków by karetka znalazła się w skrajnej pozycji, co jest rejestrowane wówczas przez krańcówke. Stanowi to punkt odniesienia, od którego odliczane są ilości skoków jakie ma wykonać silnik by ustawić karetke na konkretną pozycje. 

\begin{figure}[H]
	\centering
	\includegraphics[width=0.5\textwidth]{figures/diagram.png}
	\caption{Diagram testowej aplikacji ruchu}
	\label{fig:Diagram}
\end{figure}

%Sekcję tą można podzielić na dodatkowe podsekcje w miarę potrzeb. 
%Do tego celu nalezy wykorzystać \textit{subsection}.
%
%W przypadku, dodania istotnego fragmentu kodu należy posłużyć się środowiskiem 
%lstlisting:
%
%\begin{lstlisting}[tabsize=2]
%int foo(void){
%return 2;
%}
%\end{lstlisting}
%
%Przykładowy wzór (\ref{eq:Wzor}):
%\begin{equation}
%\label{eq:Wzor}
%\Theta = \int_t^{t+dt} \omega \, dt.	
%\end{equation}
%
%Przykładowa pozycja bibliograficzna \cite{SR01} znajduje się 
%w pliku bibliografia.bib.

% %Obecne w dokumencie do etapu II oraz III (jeśli coś zostało niezrealizowane)
% \section{Zadania niezrealizowane}

% Jeśli wszystkie zadania zostały realizowane to wówczas 
% ta sekcja powinna być usunięta w całości. W przeciwnym razie
% należy zawrzeć tutaj, jakie zadania zostały nie zrealizowane 
% oraz jaka była tego przyczyna.

%Obecne we wszystkich dokumentach
\section{Podsumowanie}

Wykonano wstępną konfiguację mikrokontrolera w programie CubeMX. Zmiany w niej będą następowały wraz z rozwojem projektu i pod kątem dostoswania ich do poszczególnych peryferiów.
DAC Audio został skonfigurowany w opcji domyślnej, USB OTG w opcji host{\_}only, która pozwoli podłączyć do mikrokontrolera zewnętrzne peryferium - dysk flash celem zapisania plików dźwiękowych.


\newpage
%\addcontentsline{toc}{section}{Bibilografia}
%\bibliography{bibliografia}
%\bibliographystyle{plabbrv}
\begin{thebibliography}{9}

\bibitem{stm}
    A. Kurczyk, \emph{Mikrokontrolery STM32 dla początkujących.}, Wydawnictwo Btc, Legionowo 2019
\bibitem{XD}
    A. France, \emph{Świat druku 3D. Przewodnik}, Wydawnictwo Helion, Gliwice 2014
\bibitem{polulu}
    Nota katalogowa A4988, \url{https://botland.com.pl/pl/index.php?controller=attachment&id_attachment=87}, 2009
\bibitem{LCD}
    Nota katalogowa PCF8574, \url{https://botland.com.pl/index.php?controller=attachment&id_attachment=210}, 1997
\bibitem{USB otg}
    Bartek, \emph{Kurs STM32 F4 – 11 – Komunikacja przez USB}, 9 sierpień 2016. Dostępne w Forbot: \url{https://forbot.pl/blog/kurs-stm32-f4-11-komunikacja-przez-usb-id13477}
\bibitem{USB flash}
    Wojtek, \emph{12 STM32F4 - CubeMx - USB podłączenie pendriva}, 26 luty 2017. Dostępne w Elektornika i Programowanie \url{https://elektronika327.blogspot.com/2017/02/12-stm32f4-cubemx-usb-podaczenie.html}
\bibitem{DAC}
    Nota aplikacyjna \emph{Audio and waveform generation using the DAC}, \url{https://www.st.com/resource/en/application_note/cd00259245-audio-and-waveform-generation-using-the-dac-in-stm32-microcontrollers-stmicroelectronics.pdf}, 2017
%\bibitem{Silnik skokowy Mineba 17PM}
%Silnik skokowy Mineba 17PM
%\url{http://cnc25.free.fr/documentation/moteurs}
%    Nota katalogowa Mineba 17PM-K, \url{https://www.eminebea.com/en/product/rotary/steppingmotor/hybrid/standard/__icsFiles/afieldfile/2017/09/29/17pm-k_1.pdf}
\bibitem{Audio playback and recording using the STM32F4DISCOVERY}
    Nota aplikacyjna \emph{Audio playback and recording},
    \url{https://www.st.com/resource/en/application_note/dm00040802-audio-playback-and-recording-using-the-stm32f4discovery-stmicroelectronics.pdf}, 2011
\bibitem{8 Channel 5V Optical Isolated Relay Module}
    Instrukcja obsługi modułu przekaźników, \url{http://www.handsontec.com/dataspecs/module/8Ch-relay.pdf}
\end{thebibliography}
\end{document}







































